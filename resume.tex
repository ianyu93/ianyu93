\documentclass[10pt, letterpaper]{article}

% Packages:
\usepackage[
    ignoreheadfoot, % set margins without considering header and footer
    top=1.5 cm, % seperation between body and page edge from the top
    bottom=1.5 cm, % seperation between body and page edge from the bottom
    left=1.5 cm, % seperation between body and page edge from the left
    right=1.5 cm, % seperation between body and page edge from the right
    footskip=1.0 cm, % seperation between body and footer
    % showframe % for debugging 
]{geometry} % for adjusting page geometry
\usepackage{titlesec} % for customizing section titles
\usepackage{tabularx} % for making tables with fixed width columns
\usepackage{array} % tabularx requires this
\usepackage[dvipsnames]{xcolor} % for coloring text
\definecolor{primaryColor}{RGB}{0, 0, 0} % define primary color
\usepackage{enumitem} % for customizing lists
\usepackage{fontawesome5} % for using icons
\usepackage{amsmath} % for math
\usepackage[
    pdftitle={Ian's CV},
    pdfauthor={Ian Yu},
    pdfcreator={LaTeX with RenderCV},
    colorlinks=true,
    urlcolor=primaryColor
]{hyperref} % for links, metadata and bookmarks
\usepackage[pscoord]{eso-pic} % for floating text on the page
\usepackage{calc} % for calculating lengths
\usepackage{bookmark} % for bookmarks
\usepackage{lastpage} % for getting the total number of pages
\usepackage{changepage} % for one column entries (adjustwidth environment)
\usepackage{paracol} % for two and three column entries
\usepackage{ifthen} % for conditional statements
\usepackage{needspace} % for avoiding page brake right after the section title
\usepackage{iftex} % check if engine is pdflatex, xetex or luatex

% Ensure that generate pdf is machine readable/ATS parsable:
\ifPDFTeX
    \input{glyphtounicode}
    \pdfgentounicode=1
    \usepackage[T1]{fontenc}
    \usepackage[utf8]{inputenc}
    \usepackage{lmodern}
\fi

\usepackage{charter}

% Some settings:
\raggedright
\AtBeginEnvironment{adjustwidth}{\partopsep0pt} % remove space before adjustwidth environment
\pagestyle{empty} % no header or footer
\setcounter{secnumdepth}{0} % no section numbering
\setlength{\parindent}{0pt} % no indentation
\setlength{\topskip}{0pt} % no top skip
\setlength{\columnsep}{0.15cm} % set column seperation
\pagenumbering{gobble} % no page numbering

\titleformat{\section}{\needspace{4\baselineskip}\bfseries\large}{}{0pt}{}[\vspace{1pt}\titlerule]

\titlespacing{\section}{
    % left space:
    -1pt
}{
    % top space:
    0.3 cm
}{
    % bottom space:
    0.2 cm
} % section title spacing

\renewcommand\labelitemi{$\vcenter{\hbox{\small$\bullet$}}$} % custom bullet points
\newenvironment{highlights}{
    \begin{itemize}[
        topsep=0.10 cm,
        parsep=0.10 cm,
        partopsep=0pt,
        itemsep=0pt,
        leftmargin=0 cm + 10pt
    ]
}{
    \end{itemize}
} % new environment for highlights


\newenvironment{highlightsforbulletentries}{
    \begin{itemize}[
        topsep=0.10 cm,
        parsep=0.10 cm,
        partopsep=0pt,
        itemsep=0pt,
        leftmargin=10pt
    ]
}{
    \end{itemize}
} % new environment for highlights for bullet entries

\newenvironment{onecolentry}{
    \begin{adjustwidth}{
        0 cm + 0.00001 cm
    }{
        0 cm + 0.00001 cm
    }
}{
    \end{adjustwidth}
} % new environment for one column entries

\newenvironment{twocolentry}[2][]{
    \onecolentry
    \def\secondColumn{#2}
    \setcolumnwidth{\fill, 4.5 cm}
    \begin{paracol}{2}
}{
    \switchcolumn \raggedleft \secondColumn
    \end{paracol}
    \endonecolentry
} % new environment for two column entries

\newenvironment{threecolentry}[3][]{
    \onecolentry
    \def\thirdColumn{#3}
    \setcolumnwidth{, \fill, 4.5 cm}
    \begin{paracol}{3}
    {\raggedright #2} \switchcolumn
}{
    \switchcolumn \raggedleft \thirdColumn
    \end{paracol}
    \endonecolentry
} % new environment for three column entries

\newenvironment{header}{
    \setlength{\topsep}{0pt}\par\kern\topsep\centering\linespread{1.5}
}{
    \par\kern\topsep
} % new environment for the header

\newcommand{\placelastupdatedtext}{% \placetextbox{<horizontal pos>}{<vertical pos>}{<stuff>}
  \AddToShipoutPictureFG*{% Add <stuff> to current page foreground
    \put(
        \LenToUnit{\paperwidth-2 cm-0 cm+0.05cm},
        \LenToUnit{\paperheight-1.0 cm}
    ){\vtop{{\null}\makebox[0pt][c]{
        \small\color{gray}\textit{Last updated in May 2025}\hspace{\widthof{Last updated in May 2025}}
    }}}%
  }%
}%

% save the original href command in a new command:
\let\hrefWithoutArrow\href

% new command for external links:


\begin{document}
    \newcommand{\AND}{\unskip
        \cleaders\copy\ANDbox\hskip\wd\ANDbox
        \ignorespaces
    }
    \newsavebox\ANDbox
    \sbox\ANDbox{$|$}

    \begin{header}
        \fontsize{25 pt}{25 pt}\selectfont Ian Yu

        \vspace{5 pt}

        \normalsize
        \mbox{Pragmatic ML Engineer in the Age of Generative AI}%
        \kern 5.0 pt%
        \AND%
        \kern 5.0 pt%
        \mbox{647 972 4689}%
        \kern 5.0 pt%
        \AND%
        \kern 5.0 pt%
        \mbox{\hrefWithoutArrow{mailto:ian.yu@arc.com.co}{ian.yu@arc.com.co}}%
        \kern 5.0 pt%
        \AND%
        \kern 5.0 pt%
        \mbox{\href{https://www.linkedin.com/in/ian-yu1/}{LinkedIn}}%
        \kern 5.0 pt%
        \AND%
        \kern 5.0 pt%
        \mbox{\href{https://github.com/ianyu93}{GitHub}}%
    \end{header}

    \vspace{5 pt - 0.3 cm}

    \section{Highlights}

    \begin{onecolentry}
        \begin{highlights}
            \item Machine Learning Engineer with 5+ years of experience designing and deploying production-grade ML systems in the eCommerce industry. Deployed 5 agentic projects to production and contributed to 2 open research initiatives on LLMs.
            \item Proficient in production-grade NLP, agentic workflows, RAG, tool-use, MCPs, OCR, MLOps, analytics data engineering, excel at designing-prototype to implement-at-scale transition
            \item Experienced in consultancy, communication with various levels of stakeholders, product thinking and strategize with users, specialized in utilizing subject matter knowledge with AI
        \end{highlights}
    \end{onecolentry}

    \section{Relevant Experience}

    \begin{twocolentry}{
        Nov 2025 – Present
    }
        \textbf{LLM Engineer}, Pixomondo (Sony Subsidiary) -- Toronto, ON\end{twocolentry}

    \vspace{0.10 cm}
    \begin{onecolentry}
        \begin{highlights}
            \item Assigned to a high impact innovation platform project with most artifacts built from scratch
            \item Built custom system discovery and registry flow for internal tools on an event-driven platform, enabling agent tool use over 20+ system services
            \item Designing and implementing event-driven internal tool use evaluation, modification suggestion, and continual learning

        \end{highlights}
    \end{onecolentry}

    \vspace{0.2 cm}

    \begin{twocolentry}{
        Apr 2024 – Present
    }
        \textbf{Machine Learning Engineer}, Aggregate Intellect, Contract -- Toronto, ON\end{twocolentry}

    \vspace{0.10 cm}
    \begin{onecolentry}
        \begin{highlights}
            \item For a RAG ChatBot client with high-fidelity requirements in legal fields, redesigned agentic workflows with 10x lower latency, 3-5x lower cost, while maintaining same output quality
            \item Architect and implemented agentic systems fora legal tech company, including document intelligence (Wordx and PDFs), RAG conversations, and memory.
        \end{highlights}
    \end{onecolentry}

    \vspace{0.2 cm}

    \begin{twocolentry}{
        Aug 2022 – Oct 2025
    }
        \textbf{Machine Learning Engineer}, RezolveAI (formerly Groupby Inc) -- Toronto, ON\end{twocolentry}

    \vspace{0.10 cm}
    \begin{onecolentry}
        \begin{highlights}
            \item Led, designed, and productionized Enrich AI, a major product with agentic workflow that manages product taxonomy, data strategy, and information enrichment, cutting full-service project timelines from weeks to hours. 
            \item Built autotuning text clustering system with custom loss and outlier handling for efficient AI reasoning reviews, deployed via Ray and Argo on GKE. Lowered orders of magnitude compute while increasing quality by 70\%
            \item Revamped AI in a patent-pending automotive fitment solution using RAG, achieving >1000 QPS and maintaining <1s latency at 99th percentile.
        \end{highlights}
    \end{onecolentry}

    \vspace{0.2 cm}

    \begin{twocolentry}{
        Apr 2021 – Jul 2022
    }
        \textbf{Data Strategy Analyst}, Groupby Inc -- Toronto, ON\end{twocolentry}

    \vspace{0.10 cm}
    \begin{onecolentry}
        \begin{highlights}
            \item Created internal packages for client-specific data enrichment analysis, boosting client satisfaction by 50\% and reducing revisions by 40\%. Automated NLP/ML processes enhancing data observability and collaboration, improving team productivity by 50\%.
            \item Introduced weakly-supervised text classification system to withhold null tasks from data annotators, saved ~\$100,000 USD troughout its lifetime.
        \end{highlights}
    \end{onecolentry}


    \section{Open Contribution}

    \begin{twocolentry}{
        2021 – 2022
    }
        \textbf{The BigScience ROOTS Corpus: A 1.6TB Composite Multilingual Dataset} \href{https://arxiv.org/abs/2303.03915}{[arxiv]}\end{twocolentry}

    \vspace{0.10 cm}
    \begin{onecolentry}
        \begin{highlights}
            \item Led a team of 10 individual contributors to minimize Personally Identifiable Information leakage. Analyzed and defined data filtering parameters to ensure corpus quality, safeness, and ethics
            \item NeurIPS 2022, Datasets and Benchmarks Track
        \end{highlights}
    \end{onecolentry}

    \vspace{0.2 cm}

    \begin{twocolentry}{
        2022 – 2022
    }
        \textbf{SantaCoder: don't reach for the stars!} \href{https://arxiv.org/abs/2301.03988}{[arxiv]}\end{twocolentry} 

    \vspace{0.10 cm}
    \begin{onecolentry}
        \begin{highlights}
            \item Advised on data annotation best practices and framework to facilitate PII detection effort for the Stack corpus
        \end{highlights}
    \end{onecolentry}

    \section{Technologies}

    \begin{onecolentry}
      \textbf{Machine Learning:} PyTorch, Transformers, Huggingface, Scikit-learn, SciPy, SpaCy, Snorkel, Vector Databases, Arrows, Polars, SageMaker, Vertex AI, Ray, Kubeflow, MLflow, DSPy
    \end{onecolentry}



    \vspace{0.2 cm}
    \begin{onecolentry}
        \textbf{Engineering:} Python, JavaScript, TypeScript, SQL, NoSQL, Bash, Linux, Docker, Kubernetes, Argo, CI/CD, gRPC, REST, GraphQL, OpenAPI, Locust, AWS, GCP
    \end{onecolentry}
  
    % \vspace{0.2 cm}
    % \begin{onecolentry}
    %     \textbf{Geospatial:} Geopandas, Shapely, turfpy, folium, PostGIS, googlemaps, geojson
    % \end{onecolentry}

    \placelastupdatedtext
\end{document}


% Backup skills based on job description:
% - LangChain, LlamaIndex, Looker, turfpy
